\documentclass[10pt,a4paper]{article}
\usepackage[pdftex]{graphicx}
\DeclareGraphicsExtensions{.pdf,.png}
\renewcommand{\d}{\mathrm{d}}
\setlength{\textwidth}{160mm}
\setlength{\oddsidemargin}{0mm}
\setlength{\evensidemargin}{0mm}

\begin{document}

\begin{center}
{\bf\large Geometry of the fine latitude strip in lcurve}\\
\end{center}

\section{The problem}

In order to speed the computation of transits with extreme radius ratios, it
helps to finely space points only along the path of the small star across the
big star. This avoids spending too long computing a million or more points in
places the transit does not affect. In lcurve, owing to the difficulties in
specifying arbitrarily distributed points, a slightly simpler approach is
taken of finely spacing points along the range of latitudes that the small
star affects. This overcooks it somewhat for inclinations below 90, but is
still a big improvement upon simply fine-gridding the entire star. This
document summarises the important equations for this problem.

I work in units of the separation. The large star is denoted number 2, 
while the small star is star 1. This is because generally lcurve is set
up to expect the small star to be the primary. The following quantities are 
defined:
\begin{enumerate}
\item[$r_1$] scaled radius of small star
\item[$r_2$] scaled radius of large star
\item[$x$, $y$] coordinates projected on the sky with 
centre of large star as the origin
\item[$i$] orbital inclination
\item[$\phi$] azimuthal coordinate for spherical polar description of surface
  of large star, defined to be zero along the meridian that runs straight down the
  visible face of the star (i.e. it moves with binary phase relative to the
  star but is fixed relative to the observer).
\item[$\theta$] latitude coordinate on large star, starting from 0 at the
  visible pole, ending at $\pi$ at the invisible pole. Poles lined up with
  orbital axis.
\item[$\lambda$] the binary phase, defined to be zero at the mid-transit point.
\end{enumerate}

Both stars are assumed spherical here. I have not tried to work out
non-spherical corrections and instead have allowed the user to supply a fudge
factor by which the latitude range can be extended if need be.

\section{Large star results}

This section gives results for latitudes and positions on the large star.

\subsection{Lines of equal latitude}
Lines of equal latitude on the large star obey the parametric equations:
\begin{eqnarray*}
x &=& r_2 \sin \theta \sin \phi,\\
y &=& r_2 \left(\sin i \cos \theta - \cos i \sin \theta \cos \phi\right),\\
\end{eqnarray*}
where the azimuthal coordinate $\phi$ is the controlling parameter.
For $\theta < \pi/2 - i$, one can see the entire line of equal latitude
so $\phi$ runs from $-\pi$ to $+\pi$, otherwise it runs from $-\phi_c$ to
$+\phi_c$ where
\begin{equation}
\cos \phi_c = - \frac{\cos \theta \cos i}{\sin \theta \sin i} .
\end{equation}

\subsection{Latitude for a given $x$, $y$ position}
The latitudes corresponding to a particular $x$, $y$ project sky position 
are given by
\begin{equation}
\cos \theta = \frac{y}{r_2} \sin i \pm \frac{\sqrt{r_2^2 - x^2 - y^2}}{r_2}
\cos i .
\end{equation}
The visible latitude corresponds to the positive root of this expression.

\section{Small star results}

For circular orbits, the small star executes an ellipse centred upon 
the large star with the projected coordinates of the centre of the small star
given by
\begin{eqnarray*}
x_0 &=& \sin \lambda,\\
y_0 &=& -\cos i \cos \lambda.
\end{eqnarray*}
As it sweeps around its orbit it maps out a swathe across the sky
with inner and outer envelopes defined by the following parametric 
equations in $\lambda$:
\begin{eqnarray*}
x &=& \sin \lambda \mp \frac{r_1 \cos i \sin \lambda}{\sqrt{\cos^2 i + \sin^2
    i \cos^2 \lambda}},\\
y &=& -\cos i \cos \lambda \pm \frac{r_1 \cos \lambda}{\sqrt{\cos^2 i + \sin^2
    i \cos^2 \lambda}}.
\end{eqnarray*}
The upper of the signs corresponds to the inner envelope (which is the upper
of the two for phases of interest to us here). The latitude of
the point where the lower/outer envelope cuts the edge of the large star defines the 
lowest latitude that needs to be finely gridded. The latitude of the upper/inner
envelope at $\lambda = 0$ defines the uppermost latitude to consider; see the figure.
\begin{figure}
\includegraphics[width=0.95\textwidth]{vlat}
\caption{Small star transiting a larger one showing lines of equal latitude on the large
star, and the envelopes of the region occulted by the small one. The upper
envelope reaches its highest latitude as it crosses the centre of the large
star while the lower envelope reaches the lowest latitude at the edge of the
large star.}
\end{figure}

\section{Implementation in lcurve}
In lcurve a binary chop is used along with the parametric equation of the
outer envelope to work out where it cuts the large star's edge. A fudge factor
is allowed to increase the upper and lower bounds derived from this procedure
to allow for possible corrections due to non-spherical effects.

\end{document}


